
\begin{abstract}

This paper describes a new programming language intended to 
make it easier to write micro-kernels for PISA.  The 
programming model is strictly at the assembly level, 
however there are high-level language constructs that 
are available.  The language is called Duct Tape, or 
simply \emph{dt} for short.  This style of programming
allows the programmer very tight control over the emitted 
instructions without the burden of hand-encoding 
each instruction.  The output of the \emph{dt} compiler
can target either the flat memory space of the 
FabScalar cores or the split memory space of the
debug cores with scratchpad rams.

\end{abstract}
