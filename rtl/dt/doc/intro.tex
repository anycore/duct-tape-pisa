\section{Introduction\label{sec:intro}}

Duct Tape (or \emph{dt}) is a new language for 
writing raw instructions using a 
pseudo-assembly syntax.  It differs from 
the GNU assembler, however, by providing 
several syntatical constructs borrowed from 
high-level programming languages such as 
loops and if statements.  The result is a 
hybrid language that is part assembly and 
part high-level.

The motivation for such a language was borne out 
in trying to debug and verify Verilog implementations
of the PISA instruction set from the FabScalar
project.  The need to carefully craft instruction
sequences to either avoid known bugs or to explicitly 
exercise corner-cases was evident, leading us to 
write micro-kernels.  Previously, to write micro-kernels, 
it was required to hand-encode instructions and 
put the encoded values into the memory using 
a series of test-bench macros that would write 
individual memory words. Writing micro-kernels 
in this manner severely restricts their sophistication.
The intent of \emph{dt} was to increase the 
productivity of the micro-kernel programmer 
while still giving tight control over the emitted 
code.

The GNU compiler~\cite{fsfgcc} (\emph{gcc}) can be used to 
write simple programs, but the programmer does not 
have control over several aspects: the registers 
that the compiler uses, the memory locations used, 
the types of instructions emitted, the library 
code that executes before the programs main() function, 
the format of the output file, 
and so on.  The GNU assembler (\emph{gas}) can be used to 
write instructions using assembly mnemonics, 
which also would have increased productivity 
(over hand-encoding instructions).  But of the 
limitations listed for \emph{gcc} above, \emph{gas}
only provides the additional control over the
registers used and types of instructions emitted.  

Contrasting \emph{dt} with \emph{gcc} and \emph{gas}, \emph{dt} allows 
the programmer very tight control over the emitted code. The
programmer can write programs strictly using assembly mnemonics, 
but can also use several high-level language constructs.  The 
\emph{dt} language allows the programmer to easily give 
names to registers and memory locations, use assignments and 
arithmetic/logic operations, use if-statements and while-loops, and 
can even include data within the program. To keep the 
instruction-level control over the emitted code, the high-level constructs
all result in specific instructions whose use is always the same.

The output of \emph{dt} can be either a checkpoint that can 
be used directly by the functional simulator of the 
FabScalar core testbench, or it can be plain-text files that 
can be read by the readmemh function 
of Verilog for the scratchpad memories of the debug FabScalar core.  
These benefits should drastically increase the 
productivity of the micro-kernel programmer and should also 
help in tracking down rare or specific RTL bugs.

The remainder of this paper covers many aspects of the \emph{dt}
toolchain and language.  This document can serve as a 
reference for the syntax and the exact code that will be 
emitted for each of the high-level language constructs.

