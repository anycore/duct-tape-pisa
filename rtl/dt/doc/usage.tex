\section{Usage\label{sec:usage}}

\begin{verbatim}

usage: ./bin/dt [flags] <file...>

flags:
       -flat            Output should be written to a
                        checkpoint (non-debug cores).  This
                        is the default.
       -scratchpad      Output should be written to two files,
                        one for the I-scratchpad and one for the
                        D-scratchpad (debug core) which can be read
                        in the testbench using the readmemh function.
                        This option requires that you have mem blocks
                        with addresses in the range of 0-256 for the
                        I-scratchpad, and 1024-1280 for the D-scratchpad.
       -fpga            Output should be written to three files,
                        in the format required by the fpga testbench,
                        one file for mem, pc, and regs (which is always
                        zeros).
       -out <file>      Output filenames will start with <file>.
                        The default is "a".
       -checking        Prints debug info (encodings, addresses, etc)
                        for parsed program to stdout.

\end{verbatim}
